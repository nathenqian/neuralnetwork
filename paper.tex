\documentclass[10pt,twocolumn,letterpaper]{article}

\usepackage{cvpr}
\usepackage{times}
\usepackage{epsfig}
\usepackage{graphicx}
\usepackage{amsmath}
\usepackage{amssymb}

% Include other packages here, before hyperref.

% If you comment hyperref and then uncomment it, you should delete
% egpaper.aux before re-running latex.  (Or just hit 'q' on the first latex
% run, let it finish, and you should be clear).
\usepackage[breaklinks=true,bookmarks=false]{hyperref}

\cvprfinalcopy % *** Uncomment this line for the final submission

\def\cvprPaperID{****} % *** Enter the CVPR Paper ID here
\def\httilde{\mbox{\tt\raisebox{-.5ex}{\symbol{126}}}}

% Pages are numbered in submission mode, and unnumbered in camera-ready
%\ifcvprfinal\pagestyle{empty}\fi
\setcounter{page}{1}
\begin{document}

%%%%%%%%% TITLE
\title{ Chinese Poem Generator from Image}

\author{Kelei Cao\\
Tsinghua University \\
Computer Science \& Technology\\
{\tt\small firstauthor@i1.org}
% For a paper whose authors are all at the same institution,
% omit the following lines up until the closing ``}''.
% Additional authors and addresses can be added with ``\and'',
% just like the second author.
% To save space, use either the email address or home page, not both
\and
Dichen Qian\\
Tsinghua University \\
Computer Science \& Technology\\
{\tt\small nathenqian@gmail.com}
}

\maketitle
%\thispagestyle{empty}

%%%%%%%%% ABSTRACT
\begin{abstract}
   	Chinese poem is very popular in China, for it's meaningful, convenience and gentle. It is very common in China that children can recite lot's of poems before primary school. 
	In Chinese parmary school, most Chinese teacher will teach children how to make sentences based on what they look. In addition, it's a way to cultivate the ability of communication for children.

	As the innovation of neural network has springed up these years, it's very interesting for computer whether they can make sentences based on an image. To make the thing even fantastic, we want to make the computer to generate the poem based on an specific image.
	
\end{abstract}

%%%%%%%%% BODY TEXT
\section{Introduction}
Deep learning has been proven in recent years to be an extremely useful tool for discriminative tasks. Through layers of linear transforms combined with nonlinearities, these systems learn to transform their input into an ideal representation across which we can draw clear decision boundaries. However It remains to be seen how this success might play out in the field of generative models. For computer, create the news based on the things it has learned must be the next generation.

In this paper, we will develop a generative models that can recognize the image, extract the feature in the image, and generate the poem based on the image. This is very basic for people, but hard for computer. In some sense, we want to explore the ability of computer.
%-------------------------------------------------------------------------
\subsection{Problem Statement}
This problem is a generative problem. The generative models should generate the poem that base on an specific image, which means that the poem should describe the image. For this purpose, the input image must contains some information and people can easily find out what the emotion it convey. The graph contains with different emotion is best for both people and computer to create the poem.

The output should be the beautiful poem that it's meaning is closed to this image.

\subsection{Plan}
\subsubsection{Image to Sentence}
The first part of this problem is to generate a sentence based on the image. This sentence must contain the detail of this image, and it can describe this image.
\subsubsection{English to Chinese}
The sentence is English, so we need to change it from English to Chinese so that we can process our next step.
\subsubsection{Sentence to Poem}
The second part of this problem is to generate the poem base on the sentence. What need to do is extracting the feature from the sentence and use them to generate the poem.

\section{Related Work}




%-------------------------------------------------------------------------



\end{document}
